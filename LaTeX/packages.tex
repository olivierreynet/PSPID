\usepackage[top=3cm,bottom=3cm,left=3.2cm,right=3.2cm,headsep=10pt,a4paper]{geometry} % marges
\usepackage{xcolor}
\definecolor{enstabGreen}{HTML}{C8D200} 	%vert  	#c8d200 
\definecolor{enstabLightGreen}{HTML}{E9ED99} 	%vert  	#c8d200 
\definecolor{enstabLightBlue}{HTML}{009EE0} %bleu clair 	#009ee0
\definecolor{enstabVeryLightBlue}{HTML}{99D8F3} %bleu clair 	#009ee0
\definecolor{enstabDarkBlue}{HTML}{005C8F}	%bleu foncé 	#005c8f
\definecolor{enstabDarkGrey}{HTML}{333333}	%gris fort 	#333333
\definecolor{enstabLightGrey}{RGB}{48,48,48}	%gris fort 	#333333
\definecolor{enstabParme}{HTML}{8878B2}		%parme 	#8878b2
\definecolor{enstabOrange}{HTML}{F18E00} 	%orange 	#f18e00
\usepackage[colorlinks=true,
        urlcolor=enstabLightBlue,
        anchorcolor=enstabDarkBlue,
        linkcolor=enstabDarkBlue,
        citecolor=enstabDarkGrey,
        pdfauthor={Johan B. C. Engelen},
        pdfkeywords={SVG; LaTeX; Inkscape},
        pdftitle={How to include an SVG image in LaTeX},
        pdfsubject={Describes how to include an SVG image easily in LaTeX using Inkscape}] {hyperref}
\usepackage{url}
\usepackage[utf8]{inputenc} % lettres accentuées
\usepackage[T1]{fontenc}    % Use 8-bit encoding that has 256 glyphs
\usepackage[frenchb]{babel} % Pour le français
\usepackage{cclicenses}     % Licences CC
\usepackage{epigraph}
\usepackage{eso-pic}        % pour une image en fond, page de titre
\usepackage{graphicx}       % Pour inclure des images
\graphicspath{{images/}}    % Où sont les images ?

\usepackage{booktabs}       % pour de jolis tableaux
\usepackage{fancyhdr}       % pour des entêtes et pieds de pages améliorés.
\usepackage{makeidx}        % requis pour faire les index
\usepackage[toc]{glossaries} %requis pour faire le glossaire