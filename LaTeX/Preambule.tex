\chapter{Préambule}
\epigraph{Le chemin est long du projet à la chose.}{Molière}

\section{Compilation du document}

Un document \LaTeX peut se compiler au travers d'un IDE (TexSutdio, TeXMaker par exemple).
Le répertoire de ce document contient également un Makefile qui permet de compiler simplement en ligne de commande. 
La fabrication du glossaire et de l'index est prise en charge dans ce Makefile. 

\section{Références internes}

Les références internes sont des renvois vers des figures, des tableaux ou des sections du rapport.
\LaTeX introduit un mécanisme simple pour établir ce genre de référence, via les commandes  \textsf{\textbackslash label} et \textsf{\textbackslash ref}. 
La première sert à définir une ancre dans le document, la seconde à la citer.
Voici par exemple une référence interne vers la section intitulée \textit{Approche Top-Down} (cf. section  \ref{sec:top-down}).
Les figures  et les tableaux  sont référencés de la même la manière (cf. figure \ref{fig:gomboc} et tableau \ref{tab:exemple}). \index{Table} \index{Figure}

\begin{table}[h]
\centering
\begin{tabular}{lll}
\toprule
\textbf{Algorithmes} & \textbf{Performance (s)} & \textbf{Gain (dB)}\\
\midrule
Algorithme 1 & 0.0003262 & 0.562 \\
Algorithme 2 & 0.0015681 & 0.910 \\
Algorithme 3 & 0.0009271 & 0.296 \\
\bottomrule
\end{tabular}
\caption{\label{tab:exemple}Performances et gains des algorithmes envisagés.}
\end{table}

\begin{figure}[h]
\centering\includegraphics[scale=0.25]{Gomboc.jpg}
\caption{\label{fig:gomboc}Gömböc : un objet homogène tridimensionnel mono-monostatique. (source : Wikipedia)}
\end{figure}


Pour insérer des entrées dans l'index, il suffit de déclarer des mots via la commande \textsf{\textbackslash index} comme suit\footnote{Évidemment, elle n'est pas visible dans le document pdf\dots Faites un tour à la fin du document !}. \index{Contexte du projet}

Pour utiliser le glossaire, il faut définir les termes dans le fichier \textsf{glossaire.tex} en utilisant la commande \textsf{\textbackslash newacronym\{label\}\{abbrv\}\{full\}}. 
Puis,  \textsf{\textbackslash gls\{label\}} permet de les utiliser dans le document. 


Par exemple, les UVs 3.4 et 4.4 sont une initiation à l'\gls{IS}.


\section{Références externes}

Les références externes sont des documents numériques, des livres, des articles, des images ou des vidéos qui ne sont pas présents dans le rapport. 
\LaTeX propose un mécanisme simple de citation.
Pour plus de détails, vous pouvez consulter les références suivantes \cite{maguis2010redigez,desgraupes2003latex,bitouze2010latex} qui sont présentent à la médiathèque de l'ENSTA Bretagne, ou celle-ci directement sur le web \cite{openclassroomLaTeX}.  

Pour citer des documents, il suffit d'appeler la commande \textsf{\textbackslash cite\{key\}} ainsi \cite{lamport1985i1}. 
La clé de citation est celle qui référence l'ouvrage dans le fichier de bibliographie intitulé   \textsf{bibliographie.bib}.
Ce fichier d'exemple contient tous les types de documents dont vous aurez besoin : livre, article de journal, références web,  rapport\dots 
Une fois insérée et compilée, la citation devient un lien dans le fichier pdf, redirigeant ainsi directement vers le détail de l'ouvrage cité dans la bibliographie située à la fin du document.
 
